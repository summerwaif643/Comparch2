\documentclass{article}


\title{RISC-V Architecture and Programming}
\author{Davide di Trocchio}

\begin{document}
    \maketitle
    \newpage

    \section{RISC-V Programming}
    \subsection{Registers}
    \subsubsection{Basic registers}
    \textbf{Register rd, t1, t2 \qquad \qquad t[0] $\leftarrow$ t[1], t[2]}
    \par
    Basic form of registers. May vary slightly.\medbreak
    
    \textbf{add t0, t1, t2 \qquad \qquad t[0] $\leftarrow$ t[1] + t[2]}
    \par
    Adds t1 and t2 to t0. \medbreak

    \textbf{addi t0, t0, integer \qquad \qquad t[0] $\leftarrow$ t[0] + integer}
    \par
    Adds an integer to a rd (t0). \medbreak

    \textbf{sub t0, t1, t2 \qquad \qquad t[0] $\leftarrow$ t[1] - t[2]}
    \par
    Subtracts t1 and t2 to t0. \medbreak

    \textbf{and t2, t1, t0}  \smallbreak
    Performs an AND operation between t1 and t0.\smallbreak
    \textbf{or t2, t1, t0}   \smallbreak
    Performs an OR operation between t1 and t0.\smallbreak
    \textbf{xor t2, t1, t0}  \smallbreak
    Performs a XOR operation between t1 and t0.\smallbreak
    f7, rs2, rs1, f3, rd, opcode. They have a slightly different type of 
    construction called \textbf{R-Type}. Ther is no NOT in the instruction set. 
    This is because we can build a not by simply writing a xor between a register 
    and a 0xffff type of instruction.\medbreak

    
    \textbf{andi t3, t0, 4}  \smallbreak
    Performs an ANDI operation between t0 and 4, puts 4 in the immediate part
    of t3.\smallbreak
    \textbf{ori t3, t0, 4}   \smallbreak
    Performs a ORI operation between t0 and 4, puts 4 in the immediate part
    of t3.\smallbreak
    \textbf{xori t3, t0, 4}  \smallbreak
    Performs a XORi operation between t0 and 4, puts 4 in the immediate part
    of t3.\smallbreak

    immediate, rs1, f3, rd, opcode. Their construction is called \textbf{I-type}.
    This is because the number passed as an argument is put in the upper part of 
    the register. 

    \subsubsection{Variables, Special Registers}
    Registers in forms of "zero" and such.

    \textbf{zero}
    \par
    Shorthand for zero value. Cannot be used in mov instruction. Check why later
    \underline{ex.} add t2, t1, zero \medbreak

    \textbf{.word integer}
    \par Create a word (variable) and assign value integer. Each number put 
    takes 4 byte and are being put in memory at the same location. Multiple 
    numbers can be put inside the \textit{.word} clause. Should always start
    at 0x10010000.\medbreak

    \textbf{.string "Hello"}
    \par Puts in memory a string of characters. Each character takes 1 byte plus 
    a special character which terminates the string.\medbreak 

    \textbf{.text}
    \par Simple label where we can put piece of code afterwards. Doesn't define 
    a variable and starts at memory location 0x00400000. \medbreak





    \subsection{Branching}
    Branch and jump on labels on different conditions. It follows a "format B" of bits,
    not like other registers we discussed earlier. Its format has rs1, rs2 and f3 like
    other popular registers, but it has two "c" parts of 5 and 7 bits each which act as
    counters to add up to the program counter. However, this counter can be interpreted
    by a label by the RARS compiler. :\medbreak
    \textbf{beq t1, t2, c(label)}
    \par "Branch If Equal", if t1 == t2, jump to label. \medbreak

    \textbf{bne t1, t2, c(label)}
    \par "Branch If Not Equal", if t1 != t2, jump to label. 

    \subsection{Loading and Saving}
    \par Loading and saving words in memory.\medbreak
    \textbf{lui t0, c}
    \par "Load Upper Immediate" loads in the upmost part of register c and
    puts 0 in all the rest. Results in 0xc0000. Loads 20 bits before.\medbreak

    \textbf{ori t1, t2, c}
    \par "OR Immediate" Puts in t1 the OR between t2 and c. It's a
    bit for bit OR and puts the result in the right hand side of the register.
    Loads 12 bits. Can be useful to put an ori and a lui to create a full custom
    bit. \medbreak

    \textbf{lw t0, c(t1) \qquad t0 $\leftarrow$ M[t1 + c]}
    \par "Load Word" loads a word from the memory. It loads an address saved in 
    memory t1, with offset c. \medbreak 

    \textbf{sw t0, c(t1) \qquad M[t1 + c] $\leftarrow$ t0 }
    \par "Save Word" saves a word in memory at offset t1+c. Uses s-type format,
    which consists of offset, rs1, f3 and an opcode. \medbreak


    \subsection{Function calls}
    Function calls to the operating system.\smallbreak 
    \textbf{ecall}
    \par "Exception call" are calls to specific system calls. There are some
    special registers used for system calls, like \textbf{a7}. Every other register
    is used for other function calls. They would be all registers from \textbf{a0 to a6}.
    Two are most used. \smallbreak
    \textbf{1} is used to print an integer.\smallbreak
    \textbf{4} is used to print a string. \smallbreak
    \textbf{10} is used to exit. Every program at the end must use this system call.
    In c, this is appended as \textit{return 0}. In more modern languages this is done
    automatically.  \smallbreak
    Also several other can be used to do other kind of stuff. But how can one use it?
    \smallbreak
    \textbf{An example: }\smallbreak
    .data\smallbreak
    \qquad .string "Hello world!"\smallbreak
    .text\smallbreak
    \qquad lui a0, 0x10010\smallbreak
    \qquad addi a7, zero, 4\smallbreak
    \qquad ecall\smallbreak
    \qquad addi a7, zero, 10\smallbreak
    \qquad ecall \smallbreak
    Ecalls are always present after adding an integer to the system call variables.
    (In this case we are looking at a7.)  


    \subsection{Registers construction}
    First seven bits are used for the \textbf{opcode}.\newline
    The \textbf{opcode} tells the basic operation of the instruction.\newline
    The \textbf{rd} (Register destination) gets the result of the operand. 
        In this case, it is t0.\newline
    The \textbf{funct3} selects a specific variant of the current operation.\newline
    The \textbf{funct7} Still to define. May occupy a bigger space to create immediate
    instructions. May contain jump instructions or general branching. \newline
    The \textbf{rs1} (Register source) is the first operand of the two registers.\newline
    The \textbf{rs2} (Register source 2 ) is same as before. They both take the same
        number of bits\newline
    This generates a very large binary string which is generally provided.
    Coded inside a map using a word (31, 0 bits).


    \subsection{Putting it all together.}
    \par \textbf{1.} Write a RARS program which takes in all numbers from 1 to 10 and stores them.
    Something like:  t0 $\leftarrow$ 1+2+3+4+5+6+7+8+9+10. \smallbreak
    \textbf{My solution:} \smallbreak
    addi t3, t3, 11\smallbreak
    add t2, zero, zero\smallbreak

    loop:\smallbreak
        \qquad add t1, t1, t2\smallbreak
        \qquad addi t2, t2, 1\smallbreak
        \qquad bne  t2, t3, loop\smallbreak
    \par All of this outputs 0x0000002d, which is 45, being the n-th triangular
    sum of 9.

\end{document}
